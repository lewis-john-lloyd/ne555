\documentclass[12pt,letterpaper]{exam}

\newcommand{\DocTitle}{Renew the DocTitle variable.}
\newcommand{\CourseNumber}{NE555}
\newcommand{\CourseName}{Nuclear Reactor Dynamics}
\newcommand{\DayTime}{TuTh 8:00-9:15}
\newcommand{\Room}{VV\&E B129}
\newcommand{\Term}{Spring 2011}
\newcommand{\Instructor}{Lewis John Lloyd}
\newcommand{\School}{University of Wisconsin - Madison}

\usepackage{listings}
\usepackage{mathtools}
\usepackage{xtab}
\usepackage{longtable}
\usepackage{array}
\usepackage{mathrsfs}
\usepackage{color}
\usepackage{multicol}
\usepackage{pdflscape}
\usepackage{pdfpages}
\usepackage{graphicx}
\usepackage{esint}
\usepackage{amsmath}
\usepackage{amssymb}
\usepackage{graphics}

\usepackage[
			pdfborder={0 0 0},
			urlcolor=cyan,
			pdftitle={\CourseNumber \DocTitle},
			pdfauthor={\Instructor}
			]{hyperref}
			
\usepackage[
			includeheadfoot,
			top=0.5in,
			bottom=0.5in,
			left=1.0in,
			right=1.0in
			]{geometry}
\setlength{\parindent}{0in}
\setlength{\parskip}{\baselineskip}

\lhead{\CourseNumber: \CourseName}
\chead{}
\rhead{\Term}
\lfoot{}
\cfoot{\thepage}
\rfoot{}
\coverlhead{\CourseNumber: \CourseName}
\coverchead{}
\coverrhead{\Term}
\coverlfoot{}
\covercfoot{\School}
\coverrfoot{}


\correctchoiceemphasis{\bfseries\color{red}}
\bracketedpoints
\pointsinmargin
%\addpoints
%\shadedsolutions

\lstset{ %
language=Matlab,                % the language of the code
basicstyle=\footnotesize,       % the size of the fonts that are used for the code
numbers=left,                   % where to put the line-numbers
numberstyle=\footnotesize,      % the size of the fonts that are used for the line-numbers
stepnumber=5,                   % the step between two line-numbers. If it's 1, each line 
                                % will be numbered
numbersep=5pt,                  % how far the line-numbers are from the code
backgroundcolor=\color{white},  % choose the background color. You must add \usepackage{color}
showspaces=false,               % show spaces adding particular underscores
showstringspaces=false,         % underline spaces within strings
showtabs=false,                 % show tabs within strings adding particular underscores
frame=single,                   % adds a frame around the code
tabsize=2,                      % sets default tabsize to 2 spaces
}

\renewcommand{\solutiontitle}{\noindent\textbf{Solution:}\par\noindent}

\newcommand{\mathsym}[1]{{}}
\newcommand{\unicode}{{}}
\newcommand{\tr}[1]{\tilde{#1}}
\newcommand{\LapTran}[1]{\mathscr{L}\left[#1\right]}
\newcommand{\InvLapTran}[1]{\mathscr{L}^{-1}\left[#1\right]}
\newcommand{\ParDer}[2]{\frac{\partial #1}{\partial #2}}
\newcommand{\Der}[2]{\frac{d\; #1}{d\;#2}}


\renewcommand{\DocTitle}{Prerequisite Quiz 02}
\noprintanswers
\addpoints

%-------------------------------------------------------------------------------------
\begin{document}
%-------------------------------------------------------------------------------------
\begin{coverpages}
\begin{center}
\textbf{\DocTitle}
\end{center}

\makebox[\textwidth]{Name:\enspace\hrulefill}
\vfill
\flushright{
\gradetable[h]
}
\end{coverpages}
%-------------------------------------------------------------------------------------
\begin{flushleft}
\begin{center}
\fbox{\fbox{\parbox{5.5in}{\centering
For multiple choice questions, circle your answer.
For short answer, please be concise.
Try to keep your answers in the space provided.
If you run out of room for an answer, continue on the back of the page.}}}
\end{center}
\vspace{0.1in}
%-------------------------------------------------------------------------------------
\begin{questions}
%-------------------------------------------------------------------------------------
\question[10]{
Given a differential control volume, $\Omega$, with a spatially and temporally varying source of energy, $\dot{q}'''$, derive an evolution equation for the time rate of change of the temperature within the control volume.
Use an integral approach to do this.
Please be explicit and explain what is occurring during each step of the derivation.

\fullwidth{\begin{solution}

\end{solution}}}
\pagebreak
%-------------------------------------------------------------------------------------
\question[10]{
Given a slab of thickness $a$ with a spatially varying energy source, $\dot{q}'''(x)$, solve the steady state heat equation subject to a constant temperature on the left face of the slab and an insulated right face.

\fullwidth{\begin{solution}

\end{solution}}}
\pagebreak
%-------------------------------------------------------------------------------------
\question[10]{Given a cylinder of radius $a$ with a constant energy source, $\dot{q}'''$, solve the steady state heat equation subject to a convective boundary condition on the outer face of the cylinder. The heat transfer coefficient is $h$ and the bulk temperature of the ambient fluid is $T_{\infty}$.

\fullwidth{\begin{solution}

\end{solution}}}
\pagebreak
%-------------------------------------------------------------------------------------
\question[10]{Describe the physical meaning of the following dimensionless variables and give their definitions in terms of physical variables: $Nu$, $Re$, $Bi$, $f$, and $Pr$. 

\fullwidth{\begin{solution}

\end{solution}}}
\pagebreak
%-------------------------------------------------------------------------------------
\question[10]{Explain the Reynolds Analogy. Points will be allocated based upon the level of detail and understanding demonstrated.

\fullwidth{\begin{solution}

\end{solution}}}
%-------------------------------------------------------------------------------------
\end{questions}\end{flushleft}

\end{document}