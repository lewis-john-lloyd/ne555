\documentclass[12pt,letterpaper]{exam}

\newcommand{\DocTitle}{Renew the DocTitle variable.}
\newcommand{\CourseNumber}{NE555}
\newcommand{\CourseName}{Nuclear Reactor Dynamics}
\newcommand{\DayTime}{TuTh 8:00-9:15}
\newcommand{\Room}{VV\&E B129}
\newcommand{\Term}{Spring 2011}
\newcommand{\Instructor}{Lewis John Lloyd}
\newcommand{\School}{University of Wisconsin - Madison}

\usepackage{listings}
\usepackage{mathtools}
\usepackage{xtab}
\usepackage{longtable}
\usepackage{array}
\usepackage{mathrsfs}
\usepackage{color}
\usepackage{multicol}
\usepackage{pdflscape}
\usepackage{pdfpages}
\usepackage{graphicx}
\usepackage{esint}
\usepackage{amsmath}
\usepackage{amssymb}
\usepackage{graphics}

\usepackage[
			pdfborder={0 0 0},
			urlcolor=cyan,
			pdftitle={\CourseNumber \DocTitle},
			pdfauthor={\Instructor}
			]{hyperref}
			
\usepackage[
			includeheadfoot,
			top=0.5in,
			bottom=0.5in,
			left=1.0in,
			right=1.0in
			]{geometry}
\setlength{\parindent}{0in}
\setlength{\parskip}{\baselineskip}

\lhead{\CourseNumber: \CourseName}
\chead{}
\rhead{\Term}
\lfoot{}
\cfoot{\thepage}
\rfoot{}
\coverlhead{\CourseNumber: \CourseName}
\coverchead{}
\coverrhead{\Term}
\coverlfoot{}
\covercfoot{\School}
\coverrfoot{}


\correctchoiceemphasis{\bfseries\color{red}}
\bracketedpoints
\pointsinmargin
%\addpoints
%\shadedsolutions

\lstset{ %
language=Matlab,                % the language of the code
basicstyle=\footnotesize,       % the size of the fonts that are used for the code
numbers=left,                   % where to put the line-numbers
numberstyle=\footnotesize,      % the size of the fonts that are used for the line-numbers
stepnumber=5,                   % the step between two line-numbers. If it's 1, each line 
                                % will be numbered
numbersep=5pt,                  % how far the line-numbers are from the code
backgroundcolor=\color{white},  % choose the background color. You must add \usepackage{color}
showspaces=false,               % show spaces adding particular underscores
showstringspaces=false,         % underline spaces within strings
showtabs=false,                 % show tabs within strings adding particular underscores
frame=single,                   % adds a frame around the code
tabsize=2,                      % sets default tabsize to 2 spaces
}

\renewcommand{\solutiontitle}{\noindent\textbf{Solution:}\par\noindent}

\newcommand{\mathsym}[1]{{}}
\newcommand{\unicode}{{}}
\newcommand{\tr}[1]{\tilde{#1}}
\newcommand{\LapTran}[1]{\mathscr{L}\left[#1\right]}
\newcommand{\InvLapTran}[1]{\mathscr{L}^{-1}\left[#1\right]}
\newcommand{\ParDer}[2]{\frac{\partial #1}{\partial #2}}
\newcommand{\Der}[2]{\frac{d\; #1}{d\;#2}}


\renewcommand{\DocTitle}{Quiz 02}
\renewcommand{\CourseName}{Nuclear Reactor Dynamics}
\renewcommand{\CourseNumber}{NE555}
\noprintanswers
\addpoints

%-------------------------------------------------------------------------------------
\begin{document}
%-------------------------------------------------------------------------------------
\begin{coverpages}
\begin{center}
\textbf{\DocTitle}
\end{center}

\makebox[\textwidth]{Name:\enspace\hrulefill}

\begin{center}
\fbox{\fbox{\parbox{5.5in}{\centering
Look and the last page for material properties.
Partial credit will be based on the quality and understandability of your work.
}}}
\end{center}

\vfill
\flushright{
\gradetable[h]
}
\end{coverpages}
%-------------------------------------------------------------------------------------
\begin{questions}
%-------------------------------------------------------------------------------------
\question[50]{
A graphite fuel pebble in a PBMR has an outer diameter of 6 [cm].
The pebble is cooled by the forced convection of helium, T$_{bulk}$ = 800 [K], with a heat transfer coefficient of 500 [$\frac{W}{m^2\cdot K}$]. The uranium containing TRISO particles are uniformly distributed throughout the inner 5 [cm] diameter sphere, yielding a uniform volumetric heat source of 30 [$\frac{MW}{m^3}$] during steady state operations.

Currently there is concern regarding EMF interference at nuclear power plants. 
Stray EMF, such as that from walkie talkie use, can induce current in some unshielded electric circuits.
At a hypothetical PBMR plant, this stray EMF has caused a control rod bank to eject, causing an instantaneous insertion of worth $\rho_o = \frac{\beta}{5}$. 
It takes the plant operators a minute to recognize the problem, cycle the control circuits, and reinsert the errant control rod bank.
The PBMR has a negative temperature coefficient of reactivity (see last page), which helps damp the power excursion.


Using a lumped parameter model for the pebble and the 1DG PRKE (do not use linearized point reactor kinetics) approximation, plot the average temperature of a fuel pebble and the total power (not volumetric power) of the pebble for ten minutes.

What is the volumetric power of the pebble after the transient has occurred?

Show your work. 

\fullwidth{\begin{solution}

\end{solution}}}
%-------------------------------------------------------------------------------------
\question[25]{
Using 1DG linearized point reactor kinetics, draw the block diagram for problem 1.
Do not combine transfer functions.
Make sure that any scaling functions are taken into account.
Explain why this is useful.
Find the transfer function for the system with the external reactivity being the input and the pebble temperature being the output.
\fullwidth{\begin{solution}

\end{solution}}}
%-------------------------------------------------------------------------------------
\question[50]{
A PWR UO$_2$ fuel pellet has a diameter of 9.3 [mm] and is insulated from the cladding by a helium gas gap. 
The zirconium cladding is 0.62 [mm] thick and has an outer diameter of 10.7 [mm].
The fuel rods are arrayed in a square lattice with a pitch-diameter ratio of 1.32.
The coolant, H$_2$O, is flowing past the fuel with a velocity of 3.5 [$\frac{m}{s}$] with a bulk temperature of 590 [K].
Use the Dittus-Boelter correlation for forced convection, $Nu = 0.023 Pr^{0.4} Re^{0.8}$, to find the heat transfer coefficient.

The fuel pellet experiences slight self-shielding, leading to an internal heat generation given by:
$$q(t,r)''' = q(t)'''\left(1+0.12(\frac{r}{R})^2\right)$$
The initial average volumetric heat generation in the fuel pellet is 300 [$\frac{MW}{m^3}$].

Due to a malfunction of the boric acid pump, there is a slight increase in the concentration of boric acid leading to a negative reactivity insertion into the system of $\rho = -195$ [pcm].

The fuel has a negative temperature coefficient of reactivity, $\alpha_{T_f} = -3.25 $ [pcm], based upon the weighted radial temperature, $\bar{T}=\displaystyle \frac{\int\limits_{0}^{R_{Fuel}}T(t,r)w(r)r dr}{\int\limits_{0}^{R_{Fuel}}w(r)r dr}$.
The weight function to use when collapsing the radial temperature is $w(r) = J_o(2.40483\frac{r}{R_{Fuel}})$.
Use the linearized PRKE assumption to model this transient.

Plot the radial temperature distribution at 10, 20, and 30 seconds.
Plot the average fuel temperature from zero to 100 seconds.

\fullwidth{\begin{solution}
\end{solution}}}

%-------------------------------------------------------------------------------------
\question[25]{

Try problem three using a lumped parameter model for the weighted radial fuel temperature.

\fullwidth{\begin{solution}
\end{solution}}}

%-------------------------------------------------------------------------------------
\end{questions}
\pagebreak
\thispagestyle{head}
\begin{center}
\textbf{Laplace Transforms}
\end{center}


\textbf{Function} \hfill \textbf{Transform}

$1$ \hfill $\displaystyle\frac{1}{s}$

$a$, a is a constant \hfill $\displaystyle\frac{a}{s}$

$\delta(t-\tau)$, $\delta$ is the Dirac Delta function \hfill $e^{-\tau\, s}$

$H(t-\tau)$, H is the Heaviside function \hfill $\displaystyle\frac{e^{-\tau\,s}}{s}$

$t\;H(t)$ \hfill $\displaystyle\frac{1}{s^2}$

$e^{at}$ \hfill $\displaystyle\frac{1}{s-a}$

$sin(a t)$ \hfill $\displaystyle\frac{a}{s^2+a^2}$

$cos(a t)$ \hfill $\displaystyle\frac{s}{s^2+a^2}$

$f(t)$ \hfill $\tilde{f}(s)$

$\displaystyle\frac{df(t)}{dt}$ \hfill $s \tilde{f}(s) - f(0)$

\pagebreak
\textbf{\underline{Kinetics Parameters}}

Use these kinetics parameters for all problems.

$\lambda$ = 0.0767194 [$\frac{1}{s}$]\\
$\beta$ = 0.00650 [-] \\
$\Lambda$ = 7.6427e-06 [s]

\textbf{\underline{Material Properties}}

\setlength{\extrarowheight}{2.0pt}
\begin{tabular}{l c c c c c}
& UO$_2$ & He & Zr & H$_2$O & Graphite\\
Thermal Conductivity [$\frac{W}{m\;K}$] & 3.0 		& 0.25 		& 17.0 		& 0.5101 	& 60.0	 \\
Heat Capacity [$\frac{J}{kg\;K}$] 		& 325.0 	& 5182.0 	& 320.9		& 5997.0	& 710.0  \\
Density [$\frac{kg}{m^3}$]				& 10970.0 	& 12.24		& 6534.0	& 688.6		& 2267.0 \\
Viscosity [$\frac{kg}{m\;s}$] 			& 			& 			& 			& 0.00008187& 		 \\
Prandtl Number [$-$] 					& 			& 			& 			& 0.9625	&        \\
$\alpha_{T_f}$ [$\frac{pcm}{K}$] *		& 			&			&			&			& -0.325
\end{tabular}

* $\alpha_{T_f}$ is the Temperature Coefficient of Reactivity in percent mili-k per delta K.

\ifprintanswers
\begin{landscape}
\end{landscape}
\fi

\end{document}