\documentclass[12pt,letterpaper]{exam}

\newcommand{\DocTitle}{Renew the DocTitle variable.}
\newcommand{\CourseNumber}{NE555}
\newcommand{\CourseName}{Nuclear Reactor Dynamics}
\newcommand{\DayTime}{TuTh 8:00-9:15}
\newcommand{\Room}{VV\&E B129}
\newcommand{\Term}{Spring 2011}
\newcommand{\Instructor}{Lewis John Lloyd}
\newcommand{\School}{University of Wisconsin - Madison}

\usepackage{listings}
\usepackage{mathtools}
\usepackage{xtab}
\usepackage{longtable}
\usepackage{array}
\usepackage{mathrsfs}
\usepackage{color}
\usepackage{multicol}
\usepackage{pdflscape}
\usepackage{pdfpages}
\usepackage{graphicx}
\usepackage{esint}
\usepackage{amsmath}
\usepackage{amssymb}
\usepackage{graphics}

\usepackage[
			pdfborder={0 0 0},
			urlcolor=cyan,
			pdftitle={\CourseNumber \DocTitle},
			pdfauthor={\Instructor}
			]{hyperref}
			
\usepackage[
			includeheadfoot,
			top=0.5in,
			bottom=0.5in,
			left=1.0in,
			right=1.0in
			]{geometry}
\setlength{\parindent}{0in}
\setlength{\parskip}{\baselineskip}

\lhead{\CourseNumber: \CourseName}
\chead{}
\rhead{\Term}
\lfoot{}
\cfoot{\thepage}
\rfoot{}
\coverlhead{\CourseNumber: \CourseName}
\coverchead{}
\coverrhead{\Term}
\coverlfoot{}
\covercfoot{\School}
\coverrfoot{}


\correctchoiceemphasis{\bfseries\color{red}}
\bracketedpoints
\pointsinmargin
%\addpoints
%\shadedsolutions

\lstset{ %
language=Matlab,                % the language of the code
basicstyle=\footnotesize,       % the size of the fonts that are used for the code
numbers=left,                   % where to put the line-numbers
numberstyle=\footnotesize,      % the size of the fonts that are used for the line-numbers
stepnumber=5,                   % the step between two line-numbers. If it's 1, each line 
                                % will be numbered
numbersep=5pt,                  % how far the line-numbers are from the code
backgroundcolor=\color{white},  % choose the background color. You must add \usepackage{color}
showspaces=false,               % show spaces adding particular underscores
showstringspaces=false,         % underline spaces within strings
showtabs=false,                 % show tabs within strings adding particular underscores
frame=single,                   % adds a frame around the code
tabsize=2,                      % sets default tabsize to 2 spaces
}

\renewcommand{\solutiontitle}{\noindent\textbf{Solution:}\par\noindent}

\newcommand{\mathsym}[1]{{}}
\newcommand{\unicode}{{}}
\newcommand{\tr}[1]{\tilde{#1}}
\newcommand{\LapTran}[1]{\mathscr{L}\left[#1\right]}
\newcommand{\InvLapTran}[1]{\mathscr{L}^{-1}\left[#1\right]}
\newcommand{\ParDer}[2]{\frac{\partial #1}{\partial #2}}
\newcommand{\Der}[2]{\frac{d\; #1}{d\;#2}}


\renewcommand{\DocTitle}{Prerequisite Quiz 01}
\noprintanswers
\addpoints

%-------------------------------------------------------------------------------------
\begin{document}
%-------------------------------------------------------------------------------------
\begin{coverpages}
\begin{center}
\textbf{\DocTitle}
\end{center}

\makebox[\textwidth]{Name:\enspace\hrulefill}
\vfill
\flushright{
\gradetable[h]
}
\end{coverpages}
%-------------------------------------------------------------------------------------
\begin{flushleft}
\begin{center}
\fbox{\fbox{\parbox{5.5in}{\centering
For multiple choice questions, circle your answer.
For short answer, please be concise.
Try to keep your answers in the space provided.
If you run out of room for an answer, continue on the back of the page.}}}
\end{center}
\vspace{0.1in}
%-------------------------------------------------------------------------------------
\begin{questions}
%-------------------------------------------------------------------------------------
\question[1]{
Which courses have you taken that focus on reactor statics (neutron diffusion, steady-state neutron flux)?
\fullwidth{\begin{solution}[0.75in]

\end{solution}}}
%-------------------------------------------------------------------------------------
\question[1]{
Which courses have you taken that focus on neutron transport theory?
\fullwidth{\begin{solution}[0.75in]

\end{solution}}}
%-------------------------------------------------------------------------------------
\question[1]{
Which courses have you taken that focus on obtaining analytic solutions to differential equations (ordinary or partial)?
\fullwidth{\begin{solution}[0.75in]

\end{solution}}}
%-------------------------------------------------------------------------------------
\question[1]{
Describe your experience with using numerical methods to solve differential equations.
\fullwidth{\begin{solution}[0.75in]

\end{solution}}}
%-------------------------------------------------------------------------------------
\question[2]{
Which of the following is a correct definition of the Laplace Transform? Circle the correct answer.\\
\begin{choices}
	\choice{$\mathscr{L}\{f(t)\}=\int\limits_{0}^{\infty}f(t)e^{-st}ds$}
	\choice{$\mathscr{L}\{f(t)\}=\int\limits_{0}^{\infty}f(t)e^{st}ds$}
	\correctchoice{$\mathscr{L}\{f(t)\}=\int\limits_{0}^{\infty}f(t)e^{-st}dt$}
	\choice{$\mathscr{L}\{f(t)\}=\int\limits_{0}^{\infty}f(t)e^{st}dt$}
\end{choices}
}

\pagebreak
%-------------------------------------------------------------------------------------
\question[4]{ Use integration by parts to evaluate the following integral. Show your steps.\\
\begin{center}
$\int\limits_{0}^{\infty}\, x\, e^{-b\,x}\,dx$
\end{center}

\fullwidth{\begin{solution}[1.5in]
\begin{eqnarray*}
\int\limits_{0}^{\infty}\, x\, e^{-b\,x}\,dx & = & \left[-\frac{x}{b}\,e^{-b\,x}\right]_0^{\infty} +\frac{1}{b}\int\limits_{0}^{\infty}\, e^{-b\,x}\,dx \\
& = & \left[-\frac{x}{b}\,e^{-b\,x}\right]_0^{\infty} +\frac{1}{b^2}\left[e^{-b\,x}\right]_{\infty}^{0} \\
& = & (0)-(0)+\frac{1}{b^2}-(0)\\
\int\limits_{0}^{\infty}\, x\, e^{-b\,x}\,dx & = & \frac{1}{b^2}
\end{eqnarray*}
\end{solution}}}
%-------------------------------------------------------------------------------------
\question[10]{ Solve the following differential equation using the Laplace Transform. a, b, and $y_0$ are constants.
\begin{center}\begin{minipage}{2in}
$\frac{dy}{dt}\,=\, a\,y+b$, \hfill $ y(0)=y_0$
\end{minipage}\end{center}

\fullwidth{\begin{solution}
\begin{eqnarray*}
\frac{dy}{dt}\,& = & \, a\,y+b\\
\LapTran{\frac{dy}{dt}}& = & a\,\LapTran{y}+b\,\LapTran{1}\\
s\,\tr{y}-y_0&=&a\,\tr{y}+\frac{b}{s}\\
\tr{y}\left(s-a\right)&=&y_0+\frac{b}{s}\\
\tr{y} & = & \frac{y_0}{s-a}+\frac{b}{(s)(s-a)}\\
\InvLapTran{\tr{y}}& = & y_0\,\InvLapTran{\frac{1}{s-a}}+b\,\InvLapTran{\frac{1}{(s)(s-a)}}\\
y(t) & = & y_0\,e^{a\,t}-\frac{b}{a}+\frac{b}{a}\,e^{a\,t}\\
y(t) & = & \left(y_0+\frac{b}{a}\right)\,e^{a\,t}-\frac{b}{a}
\end{eqnarray*}
\end{solution}}}
%-------------------------------------------------------------------------------------
\pagebreak
\question[5]{Write down a differential equation for the time-dependent, continuous-energy, neutron diffusion equation in a multiplying media, without a source term.
\fullwidth{\begin{solution}[1.5in]
$\displaystyle\frac{1}{v}\frac{\partial \phi}{\partial t} = 
-\nabla\cdot(-D\nabla\phi) 
-\Sigma_t\phi
+\int\limits_0^{\infty}\Sigma_s(E'\rightarrow E)\phi(E')dE'
+\chi(E)\int\limits_0^{\infty}\nu(E')\Sigma_f(E')\phi(E')dE'$
\end{solution}}
}
%-------------------------------------------------------------------------------------
\question[15]{Briefly describe how the time-dependent equation from above relates to the critical reactor problem.
Make sure to mention what role "k" plays in this formulation.
\fullwidth{\begin{solution}
The time-dependent neutron diffusion equation represents the transient behaviour of the neutron flux within a reactor.
In order for the reactor to be in a steady-state configuration, the following equation needs to hold:
$$\displaystyle
-\nabla\cdot(D\nabla\phi) 
+\Sigma_t\phi
-\int\limits_0^{\infty}\Sigma_s(E'\rightarrow E)\phi(E')dE'
=
\frac{1}{k}\chi(E)\int\limits_0^{\infty}\nu(E')\Sigma_f(E')\phi(E')dE'
$$

Recall from NE405 that the fission source, the RHS, is multiplied by a $\displaystyle \frac{1}{k}$.
This factor, $k$, reflects the equilibrium between non-fission processes and neutron production. 


If $k = 1$, the reactor is "critical," which indicates that the neutron population of the reactor is steady in time.\\
If $k < 1$, the reactor is "sub-critical," which indicates that the neutron population of the reactor is decreasing in time.\\
If $k < 1$, the reactor is "super-critical," which indicates that the neutron population of the reactor is increasing in time.\\ 

\end{solution}}}
%-------------------------------------------------------------------------------------
\end{questions}\end{flushleft}
\pagebreak
\thispagestyle{head}
\begin{center}
\textbf{Laplace Transforms}
\end{center}


\textbf{Function} \hfill \textbf{Transform}

$1$ \hfill $\displaystyle\frac{1}{s}$

$a$, a is a constant \hfill $\displaystyle\frac{a}{s}$

$\delta(t-\tau)$, $\delta$ is the Dirac Delta function \hfill $e^{-\tau\, s}$

$H(t-\tau)$, H is the Heaviside function \hfill $\displaystyle\frac{e^{-\tau\,s}}{s}$

$t\;H(t)$ \hfill $\displaystyle\frac{1}{s^2}$

$e^{at}$ \hfill $\displaystyle\frac{1}{s-a}$

$sin(a t)$ \hfill $\displaystyle\frac{a}{s^2+a^2}$

$cos(a t)$ \hfill $\displaystyle\frac{s}{s^2+a^2}$

$f(t)$ \hfill $\tilde{f}(s)$

$\displaystyle\frac{df(t)}{dt}$ \hfill $s \tilde{f}(s) - f(0)$









\end{document}